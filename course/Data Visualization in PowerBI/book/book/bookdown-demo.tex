% Options for packages loaded elsewhere
\PassOptionsToPackage{unicode}{hyperref}
\PassOptionsToPackage{hyphens}{url}
%
\documentclass[
]{book}
\usepackage{lmodern}
\usepackage{amssymb,amsmath}
\usepackage{ifxetex,ifluatex}
\ifnum 0\ifxetex 1\fi\ifluatex 1\fi=0 % if pdftex
  \usepackage[T1]{fontenc}
  \usepackage[utf8]{inputenc}
  \usepackage{textcomp} % provide euro and other symbols
\else % if luatex or xetex
  \usepackage{unicode-math}
  \defaultfontfeatures{Scale=MatchLowercase}
  \defaultfontfeatures[\rmfamily]{Ligatures=TeX,Scale=1}
\fi
% Use upquote if available, for straight quotes in verbatim environments
\IfFileExists{upquote.sty}{\usepackage{upquote}}{}
\IfFileExists{microtype.sty}{% use microtype if available
  \usepackage[]{microtype}
  \UseMicrotypeSet[protrusion]{basicmath} % disable protrusion for tt fonts
}{}
\makeatletter
\@ifundefined{KOMAClassName}{% if non-KOMA class
  \IfFileExists{parskip.sty}{%
    \usepackage{parskip}
  }{% else
    \setlength{\parindent}{0pt}
    \setlength{\parskip}{6pt plus 2pt minus 1pt}}
}{% if KOMA class
  \KOMAoptions{parskip=half}}
\makeatother
\usepackage{xcolor}
\IfFileExists{xurl.sty}{\usepackage{xurl}}{} % add URL line breaks if available
\IfFileExists{bookmark.sty}{\usepackage{bookmark}}{\usepackage{hyperref}}
\hypersetup{
  pdftitle={Data Visualization in PowerBI},
  pdfauthor={Mohamed Kassem},
  hidelinks,
  pdfcreator={LaTeX via pandoc}}
\urlstyle{same} % disable monospaced font for URLs
\usepackage{color}
\usepackage{fancyvrb}
\newcommand{\VerbBar}{|}
\newcommand{\VERB}{\Verb[commandchars=\\\{\}]}
\DefineVerbatimEnvironment{Highlighting}{Verbatim}{commandchars=\\\{\}}
% Add ',fontsize=\small' for more characters per line
\usepackage{framed}
\definecolor{shadecolor}{RGB}{248,248,248}
\newenvironment{Shaded}{\begin{snugshade}}{\end{snugshade}}
\newcommand{\AlertTok}[1]{\textcolor[rgb]{0.94,0.16,0.16}{#1}}
\newcommand{\AnnotationTok}[1]{\textcolor[rgb]{0.56,0.35,0.01}{\textbf{\textit{#1}}}}
\newcommand{\AttributeTok}[1]{\textcolor[rgb]{0.77,0.63,0.00}{#1}}
\newcommand{\BaseNTok}[1]{\textcolor[rgb]{0.00,0.00,0.81}{#1}}
\newcommand{\BuiltInTok}[1]{#1}
\newcommand{\CharTok}[1]{\textcolor[rgb]{0.31,0.60,0.02}{#1}}
\newcommand{\CommentTok}[1]{\textcolor[rgb]{0.56,0.35,0.01}{\textit{#1}}}
\newcommand{\CommentVarTok}[1]{\textcolor[rgb]{0.56,0.35,0.01}{\textbf{\textit{#1}}}}
\newcommand{\ConstantTok}[1]{\textcolor[rgb]{0.00,0.00,0.00}{#1}}
\newcommand{\ControlFlowTok}[1]{\textcolor[rgb]{0.13,0.29,0.53}{\textbf{#1}}}
\newcommand{\DataTypeTok}[1]{\textcolor[rgb]{0.13,0.29,0.53}{#1}}
\newcommand{\DecValTok}[1]{\textcolor[rgb]{0.00,0.00,0.81}{#1}}
\newcommand{\DocumentationTok}[1]{\textcolor[rgb]{0.56,0.35,0.01}{\textbf{\textit{#1}}}}
\newcommand{\ErrorTok}[1]{\textcolor[rgb]{0.64,0.00,0.00}{\textbf{#1}}}
\newcommand{\ExtensionTok}[1]{#1}
\newcommand{\FloatTok}[1]{\textcolor[rgb]{0.00,0.00,0.81}{#1}}
\newcommand{\FunctionTok}[1]{\textcolor[rgb]{0.00,0.00,0.00}{#1}}
\newcommand{\ImportTok}[1]{#1}
\newcommand{\InformationTok}[1]{\textcolor[rgb]{0.56,0.35,0.01}{\textbf{\textit{#1}}}}
\newcommand{\KeywordTok}[1]{\textcolor[rgb]{0.13,0.29,0.53}{\textbf{#1}}}
\newcommand{\NormalTok}[1]{#1}
\newcommand{\OperatorTok}[1]{\textcolor[rgb]{0.81,0.36,0.00}{\textbf{#1}}}
\newcommand{\OtherTok}[1]{\textcolor[rgb]{0.56,0.35,0.01}{#1}}
\newcommand{\PreprocessorTok}[1]{\textcolor[rgb]{0.56,0.35,0.01}{\textit{#1}}}
\newcommand{\RegionMarkerTok}[1]{#1}
\newcommand{\SpecialCharTok}[1]{\textcolor[rgb]{0.00,0.00,0.00}{#1}}
\newcommand{\SpecialStringTok}[1]{\textcolor[rgb]{0.31,0.60,0.02}{#1}}
\newcommand{\StringTok}[1]{\textcolor[rgb]{0.31,0.60,0.02}{#1}}
\newcommand{\VariableTok}[1]{\textcolor[rgb]{0.00,0.00,0.00}{#1}}
\newcommand{\VerbatimStringTok}[1]{\textcolor[rgb]{0.31,0.60,0.02}{#1}}
\newcommand{\WarningTok}[1]{\textcolor[rgb]{0.56,0.35,0.01}{\textbf{\textit{#1}}}}
\usepackage{longtable,booktabs}
% Correct order of tables after \paragraph or \subparagraph
\usepackage{etoolbox}
\makeatletter
\patchcmd\longtable{\par}{\if@noskipsec\mbox{}\fi\par}{}{}
\makeatother
% Allow footnotes in longtable head/foot
\IfFileExists{footnotehyper.sty}{\usepackage{footnotehyper}}{\usepackage{footnote}}
\makesavenoteenv{longtable}
\usepackage{graphicx,grffile}
\makeatletter
\def\maxwidth{\ifdim\Gin@nat@width>\linewidth\linewidth\else\Gin@nat@width\fi}
\def\maxheight{\ifdim\Gin@nat@height>\textheight\textheight\else\Gin@nat@height\fi}
\makeatother
% Scale images if necessary, so that they will not overflow the page
% margins by default, and it is still possible to overwrite the defaults
% using explicit options in \includegraphics[width, height, ...]{}
\setkeys{Gin}{width=\maxwidth,height=\maxheight,keepaspectratio}
% Set default figure placement to htbp
\makeatletter
\def\fps@figure{htbp}
\makeatother
\setlength{\emergencystretch}{3em} % prevent overfull lines
\providecommand{\tightlist}{%
  \setlength{\itemsep}{0pt}\setlength{\parskip}{0pt}}
\setcounter{secnumdepth}{5}
\usepackage{booktabs}
\usepackage{amsthm}
\makeatletter
\def\thm@space@setup{%
  \thm@preskip=8pt plus 2pt minus 4pt
  \thm@postskip=\thm@preskip
}
\makeatother
\usepackage[]{natbib}
\bibliographystyle{apalike}

\title{Data Visualization in PowerBI}
\author{Mohamed Kassem}
\date{2021-04-27}

\begin{document}
\maketitle

{
\setcounter{tocdepth}{1}
\tableofcontents
}
\hypertarget{intorduction}{%
\chapter{Intorduction}\label{intorduction}}

This course is intended for educational purposes \& preparing Data analysts for Exam DA-100: Analyzing Data with Microsoft Power BI : \textbf{Creating reports}

\begin{itemize}
\item
  All the needed resources to follow along can be found here \url{https://github.com/Mkassem16/NycTaxiPBI}.
\item
  Creating basic PowerBI reports knowledge is prerequisite for this course.
\end{itemize}

\textbf{By the end of this course Data Analysts should be able to:}\\
+ Customize Report pages.\\
+ Decide for appropriate visualizations type.\\
+ To utilize PowerBI native visualizations.\\
+ Format and configure PowerBI native visuals.\\
+ Configure conditional formatting.\\
+ Import their own custom R or Python Visuals.\\
+ Utilize Slicers and Filters.\\
+ Adjust visuals for accessibility.\\
+ Configure automatic page refresh.\\
+ Create paginated reports.

\hypertarget{setting-and-objectives}{%
\chapter{Setting and Objectives}\label{setting-and-objectives}}

\hypertarget{setting}{%
\section{Setting}\label{setting}}

The New York City Taxi and Limousine Commission (TLC), created in 1971, is the agency responsible for licensing and regulating New York City's Medallion (Yellow) taxi cabs. Over 200,000 TLC licensees complete approximately 1,000,000 trips each day.

For the sake of educational purposes Data Analysts should think of an Imaginary situation where NYC TLC has asked them to prepare a PowerBI report for their c-suite executives and decision managers to better under stand their monthly operations in general and track their daily and monthly KPIs.

\hypertarget{objectives}{%
\section{Objectives}\label{objectives}}

\hypertarget{general}{%
\subsection{General}\label{general}}

\begin{itemize}
\tightlist
\item
  Creating an easy to read and understand descriptive dashboard.
\item
  Quick one look dynamic status updates.
\item
  Tracking daily and weekly KPIs.
\item
  Understanding revenue breakdown.
\end{itemize}

\hypertarget{requested}{%
\subsection{Requested}\label{requested}}

\begin{itemize}
\tightlist
\item
  Is there a specific day of week that NYC TLC should deploy more licensees?
\item
  Is there a specific time of day that NYC TLC should deploy more licensees?
\end{itemize}

\hypertarget{extra}{%
\subsection{Extra}\label{extra}}

\begin{itemize}
\tightlist
\item
  Mining for relations between trip duration, distance and fare.
\end{itemize}

\hypertarget{data}{%
\chapter{Data}\label{data}}

Data is publicly available at \url{https://www1.nyc.gov/site/tlc/about/tlc-trip-record-data.page}

Please Find Data catalog explaining columns and categories here: \url{https://github.com/Mkassem16/NycTaxiPBI/blob/main/data/data_dictionary_trip_records_yellow.pdf}

For the purpose of building a demonstration dashboard, I extracted a sample of 10,000 rows of 2020 data. This sample dataset is hosted publicly on Google Cloud storage and can be queried directly from PowerBi.
\url{https://storage.googleapis.com/powerbi_datacamp/nyc_taxi_db.csv}

\begin{Shaded}
\begin{Highlighting}[]
\KeywordTok{glimpse}\NormalTok{(df)}
\end{Highlighting}
\end{Shaded}

\begin{verbatim}
## Rows: 10,018
## Columns: 17
## $ vendor_id           <dbl> 1, 1, 2, 1, 2, 1, 4, 2, 1, 2, 1, 2, 2, 2, 2, 2,...
## $ pickup_datetime     <dttm> 2021-03-27 12:47:16, 2021-06-10 19:02:02, 2021...
## $ dropoff_datetime    <dttm> 2021-03-27 13:39:54, 2021-06-10 19:31:53, 2021...
## $ passenger_count     <dbl> 1, 1, 1, 1, 2, 1, 1, 1, 1, 1, 1, 1, 1, 2, 1, 2,...
## $ trip_distance       <dbl> 2.70, 15.10, 7.92, 6.50, 6.44, 10.00, 7.24, 36....
## $ rate_code           <dbl> 1, 1, 1, 1, 1, 1, 1, 5, 1, 1, 1, 1, 1, 1, 1, 1,...
## $ store_and_fwd_flag  <chr> "N", "N", "N", "N", "N", "N", "N", "N", "N", "N...
## $ payment_type        <dbl> 1, 1, 1, 1, 1, 1, 1, 1, 1, 1, 1, 1, 1, 1, 1, 1,...
## $ fare_amount         <dbl> 29.0, 42.0, 26.0, 22.5, 24.5, 29.0, 22.0, 83.5,...
## $ extra               <dbl> 0.0, 0.0, 0.5, 0.5, 0.5, 0.5, 0.5, 0.0, 0.0, 0....
## $ mta_tax             <dbl> 0.5, 0.5, 0.5, 0.5, 0.5, 0.5, 0.5, 0.0, 0.5, 0....
## $ tip_amount          <dbl> 5.95, 12.10, 5.46, 4.75, 3.87, 6.05, 4.66, 21.5...
## $ tolls_amount        <dbl> 0.00, 5.76, 0.00, 0.00, 0.00, 0.00, 0.00, 23.76...
## $ imp_surcharge       <dbl> 0.3, 0.3, 0.3, 0.3, 0.3, 0.3, 0.3, 0.3, 0.3, 0....
## $ total_amount        <dbl> 35.75, 60.66, 32.76, 28.55, 29.67, 36.35, 27.96...
## $ pickup_location_id  <dbl> 68, 138, 261, 262, 261, 100, 7, 132, 264, 170, ...
## $ dropoff_location_id <dbl> 162, 88, 41, 231, 162, 127, 53, 265, 264, 236, ...
\end{verbatim}

\hypertarget{dataset}{%
\chapter{Dataset}\label{dataset}}

\emph{Dataset} in PowerBI vocabulary is: the set of queries that the user will perform in order to import all of his data tables and all the transformations that will take place before data can be ready for the model and visualization.

\textbf{We will perform 2 queries to create 2 data tables:}

\begin{itemize}
\tightlist
\item
  Table 1 (\textbf{nyc\_taxi\_21}) : will include NYC TLC data. \url{https://github.com/Mkassem16/NycTaxiPBI/blob/main/queries/nyc_taxi_21.pq}
\item
  Table 2 (\textbf{Calendar}) : Calendar table that will control the date dimension
  \url{https://github.com/Mkassem16/NycTaxiPBI/blob/main/queries/Calendar.pq}
\end{itemize}

\hypertarget{step-1}{%
\section{Step 1}\label{step-1}}

Getting data through Blank query
\includegraphics{assets/get_data.png}

\hypertarget{step-2}{%
\section{Step 2}\label{step-2}}

Choose Advanced editor
\includegraphics{assets/get_data2.png}

\hypertarget{step-3}{%
\section{Step 3}\label{step-3}}

Get the query of Table 1 from the provided link and paste it here instead of
what is already there
\includegraphics{assets/query_1.png}
\#\# Step 4

Make sure that query name is \textbf{nyc\_taxi\_21}
\includegraphics{assets/query_1b.png}

\hypertarget{step-5}{%
\section{Step 5}\label{step-5}}

Get the query of Table 2 from the provided link and paste it here instead of
what is already there
\includegraphics{assets/query_2.png}

\hypertarget{step-6}{%
\section{Step 6}\label{step-6}}

Make sure that query name is \textbf{Calendar}
\includegraphics{assets/query_2b.png}

\hypertarget{step-7}{%
\section{Step 7}\label{step-7}}

Double check queries names as its important for Measures created through DAX,
Then click close \& apply
\includegraphics{assets/get_data3.png}

\hypertarget{datamodel}{%
\chapter{DataModel}\label{datamodel}}

\emph{Data Modeling} in PowerBI vocabulary is: the step of creating relations between tables before proceeding and building slicers and filters on our visuals we need to make sure that the relationships are with the right direction, cardinality and that they are active.

\textbf{We will create 1 relation between Calendar table and the data table}

\hypertarget{step-1-1}{%
\section{Step 1}\label{step-1-1}}

Navigate to data modeling view
\includegraphics{assets/datamodel1.png}

\hypertarget{step-2-1}{%
\section{Step 2}\label{step-2-1}}

Click on manage relations
\includegraphics{assets/datamodel2.png}

\hypertarget{step-3-1}{%
\section{Step 3}\label{step-3-1}}

Create new relation and make sure of the following:

\begin{itemize}
\tightlist
\item
  Select the right tables\\
\item
  Select the right columns (by clicking on the column name in the display)\\
\item
  Select the right Cardinality (usually with Calendar tables its One to many)\\
\item
  Select the right direction (\textbf{Calendar} table to filter \textbf{nyc\_tax\_21} table)\\
  \includegraphics{assets/datamodel3.png}
\end{itemize}

\hypertarget{report-page}{%
\chapter{Report page}\label{report-page}}

We can start customizing report page by clicking on the format icon of the Visualization pane without choosing any visual.

\begin{figure}
\centering
\includegraphics{assets/page_1.png}
\caption{format}
\end{figure}

\textbf{Options include:}

\hypertarget{information}{%
\section{information}\label{information}}

\hypertarget{size}{%
\section{Size}\label{size}}

\hypertarget{background}{%
\section{Background}\label{background}}

\hypertarget{alignment}{%
\section{Alignment}\label{alignment}}

\hypertarget{wallpaper}{%
\section{Wallpaper}\label{wallpaper}}

\hypertarget{guide}{%
\chapter{Guide}\label{guide}}

Figure

\begin{Shaded}
\begin{Highlighting}[]
\KeywordTok{par}\NormalTok{(}\DataTypeTok{mar =} \KeywordTok{c}\NormalTok{(}\DecValTok{4}\NormalTok{, }\DecValTok{4}\NormalTok{, }\FloatTok{.1}\NormalTok{, }\FloatTok{.1}\NormalTok{))}
\KeywordTok{plot}\NormalTok{(pressure, }\DataTypeTok{type =} \StringTok{'b'}\NormalTok{, }\DataTypeTok{pch =} \DecValTok{19}\NormalTok{)}
\end{Highlighting}
\end{Shaded}

\begin{figure}

{\centering \includegraphics[width=0.8\linewidth]{bookdown-demo_files/figure-latex/nice-fig-1} 

}

\caption{Here is a nice figure!}\label{fig:nice-fig}
\end{figure}

Table

\begin{Shaded}
\begin{Highlighting}[]
\NormalTok{knitr}\OperatorTok{::}\KeywordTok{kable}\NormalTok{(}
  \KeywordTok{head}\NormalTok{(iris, }\DecValTok{20}\NormalTok{), }\DataTypeTok{caption =} \StringTok{'Here is a nice table!'}\NormalTok{,}
  \DataTypeTok{booktabs =} \OtherTok{TRUE}
\NormalTok{)}
\end{Highlighting}
\end{Shaded}

\begin{table}

\caption{\label{tab:nice-tab}Here is a nice table!}
\centering
\begin{tabular}[t]{rrrrl}
\toprule
Sepal.Length & Sepal.Width & Petal.Length & Petal.Width & Species\\
\midrule
5.1 & 3.5 & 1.4 & 0.2 & setosa\\
4.9 & 3.0 & 1.4 & 0.2 & setosa\\
4.7 & 3.2 & 1.3 & 0.2 & setosa\\
4.6 & 3.1 & 1.5 & 0.2 & setosa\\
5.0 & 3.6 & 1.4 & 0.2 & setosa\\
\addlinespace
5.4 & 3.9 & 1.7 & 0.4 & setosa\\
4.6 & 3.4 & 1.4 & 0.3 & setosa\\
5.0 & 3.4 & 1.5 & 0.2 & setosa\\
4.4 & 2.9 & 1.4 & 0.2 & setosa\\
4.9 & 3.1 & 1.5 & 0.1 & setosa\\
\addlinespace
5.4 & 3.7 & 1.5 & 0.2 & setosa\\
4.8 & 3.4 & 1.6 & 0.2 & setosa\\
4.8 & 3.0 & 1.4 & 0.1 & setosa\\
4.3 & 3.0 & 1.1 & 0.1 & setosa\\
5.8 & 4.0 & 1.2 & 0.2 & setosa\\
\addlinespace
5.7 & 4.4 & 1.5 & 0.4 & setosa\\
5.4 & 3.9 & 1.3 & 0.4 & setosa\\
5.1 & 3.5 & 1.4 & 0.3 & setosa\\
5.7 & 3.8 & 1.7 & 0.3 & setosa\\
5.1 & 3.8 & 1.5 & 0.3 & setosa\\
\bottomrule
\end{tabular}
\end{table}

reference
Chapter \ref{methods}

R

\begin{Shaded}
\begin{Highlighting}[]
\KeywordTok{install.packages}\NormalTok{(}\StringTok{"bookdown"}\NormalTok{)}
\CommentTok{# or the development version}
\CommentTok{# devtools::install_github("rstudio/bookdown")}
\end{Highlighting}
\end{Shaded}

link
\url{https://yihui.name/tinytex/}.

bold
\textbf{bookdown}

inclined
\emph{sample}

  \bibliography{book.bib,packages.bib}

\end{document}
